%test.tex
\documentclass[twoside=false]{buptreport}
\reportsetup{%
    abstractpage = true,    %生成摘要页
    titleinabstract = false,%生成摘要页里的标题
    toc = true,             %生成目录
    pagehead = 阿司匹林简介  %设置页眉内容
}

%======测试文档语句======
\usepackage{zhlipsum}
\newcounter{aspr}
\setcounter{aspr}{0}
\newcommand{\aspirin}{\stepcounter{aspr}\zhlipsum[\theaspr][name=aspirin]}
%======测试文档语句======

\begin{document}
    \makecover{cover/cover.pdf}
    \title{阿司匹林简介}
    \begin{abstract}
        \qquad 本文是基于\LaTeX{}的论文模板,其要求大体上参考
       {《北京邮电大学本科毕业设计(论文)工作细则》}
        ,在本文中将主要介绍本模板的使用方法以及代码构成。
        为了使得行文充实,在正文多出将会使用{\ttfamily zhlipsum}
        宏包进行填充。

        \aspirin
        
    \end{abstract}
    \begin{keywords}
        论文排版;\LaTeX{};阿司匹林
    \end{keywords}
    \maketitle

    \section{模板介绍}
    这份模板由北京邮电大学信息与通信学院通信工程专业的刘淙溪设计,
    代码主要参考为清华大学的学位论文模板。
    参考书目为胡伟老师的两本书\cite{1,2}。

    列表测试
    \begin{enumerate}[label=\chinese*、]
        \item 第一条
        \begin{enumerate}[label=\arabic* .]
            \item 小一条
            \item 小二条
            \begin{enumerate}[label=\alph* .]
                \item 小小一条
                \item 小小二条
            \end{enumerate}
            \item 小三条
        \end{enumerate} 
        \item 第二条
        \item 第三条
        \item \aspirin
    \end{enumerate}

    \section{阿司匹林第一部分}          \aspirin
        \subsection{第一小部分}         \aspirin
        \subsection{第二小部分}         \aspirin
            \subsubsection{第一段}      \aspirin 
            \paragraph{四级标题}        \aspirin
            \paragraph{四级标题}        \aspirin
            \subsubsection{第二段}      \aspirin
            \subsubsection{第三段}      \aspirin
\ifodd\value{subsubsection}%subsubsection 这里使用来测试双页模式下目录断页情况的
    \section{阿司匹林第二部分}
        \subsection{第一小部分}         \aspirin
            \subsubsection{第一、二段}  \aspirin\aspirin
            \subsubsection{第三段}      \aspirin
            \subsubsection{第四段}      \aspirin
        \subsection{第二小部分}         \aspirin
            \subsubsection{第一段}      \aspirin
            \subsubsection{第二段}      \aspirin
            \subsubsection{第三段}      \aspirin
        \subsection{第三小部分}         \aspirin
            \subsubsection{第一段}      \aspirin
            \subsubsection{第二段}      \aspirin
\fi
    \section{阿司匹林第三部分}
        \subsection{第一小部分}         \aspirin
            \subsubsection{第一段}      \aspirin
        \subsection{第二小部分}         
            \subsubsection{第一段}      \aspirin
    \section{阿司匹林第四部分}
        \subsection{第一小部分}         
            \subsubsection{第一段}      \aspirin
        \subsection{第二小部分}         
            \subsubsection{第一段}      \aspirin

    \begin{buptbibliography}
        \bibitem{1} 胡伟.LaTeXe完全学习手册.第二版.清华大学出版社.2013
        \bibitem{2} 胡伟.LaTeXe文类和宏包学习手册.清华大学出版社.2017
    \end{buptbibliography}

    \acknowledgement
    \aspirin
    \aspirin

    \begin{appendix}
    
    \section{该吃药了}       \aspirin
        \subsection{什么药}      \aspirin
            \subsubsection{阿司匹林} \aspirin

    \section{所以}          \aspirin
        \subsection{你应该} \aspirin
            \subsubsection{对阿司匹林} \aspirin
            \subsubsection{有所了解}   \aspirin
        \subsection{了吧}   \aspirin
    \section{余下全文看个够}
        \stepcounter{aspr}
        \zhlipsum[\theaspr-][name=aspirin]
    \end{appendix}
\end{document}